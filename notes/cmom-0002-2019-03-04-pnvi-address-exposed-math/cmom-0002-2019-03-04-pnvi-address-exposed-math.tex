
\documentclass[acmsmall,review,screen]{acmart}\settopmatter{printfolios=true,printccs=false,printacmref=false}

\citestyle{acmauthoryear}

\newif\ifWGfourteennumber
\WGfourteennumberfalse


\usepackage[normalem]{ulem}
\usepackage{wrapfig}
\usepackage{minibox}

%% Some recommended packages.
\usepackage{booktabs}   %% For formal tables:
                        %% http://ctan.org/pkg/booktabs
\usepackage{subcaption} %% For complex figures with subfigures/subcaptions
                        %% http://ctan.org/pkg/subcaption
\usepackage{verbatim}


\newif\ifmycopyright\mycopyrightfalse
%\mycopyrighttrue
\setcopyright{none}

\bibliographystyle{ACM-Reference-Format}
%% Citation style
%\citestyle{acmauthoryear}  %% For author/year citations
%\citestyle{acmnumeric}     %% For numeric citations
%\setcitestyle{nosort}      %% With 'acmnumeric', to disable automatic
                            %% sorting of references within a single citation;
                            %% e.g., \cite{Smith99,Carpenter05,Baker12}
                            %% rendered as [14,5,2] rather than [2,5,14].
%\setcitesyle{nocompress}   %% With 'acmnumeric', to disable automatic
                            %% compression of sequential references within a
                            %% single citation;
                            %% e.g., \cite{Baker12,Baker14,Baker16}
                            %% rendered as [2,3,4] rather than [2-4].


%%%%%%%%%%%%%%%%%%%%%%%%%%%%%%%%%%%%%%%%%%%%%%%%%%%%%%%%%%%%%%%%%%%%%%
\usepackage[utf8]{inputenc}
\usepackage[T1]{fontenc}
\usepackage{amsmath,amsthm,amssymb,bm,mathtools}

\usepackage[all]{xy}

% these are useful comments, maybe we should make them footnotes?
\renewcommand{\comment}[2][comment]{{\color{blue}\bf[#1: #2]}}
% uncomment the following line to remove comments
 \renewcommand{\comment}[2][comment]{}

\newcommand{\TODO}[1]{{\color{red}\bf[TODO: #1]}}
\definecolor{VGcolor}{rgb}{0.9 0.2 0.0}
\newcommand{\VG}[1]{{\color{VGcolor}\bf[VG: #1]}}
\definecolor{Kcolour}{rgb}{0.87451 0.5 0.0}
\newcommand{\K}[1]{{\color{Kcolour}\bf[K: #1]}}
%\renewcommand{\TODO}[1]{}

\newcommand{\myparagraph}[1]{\vspace{0.5\baselineskip}\par\noindent{\normalsize\bfseries{#1}}\quad}
\newcommand{\myqs}[1]{\vspace{0.2\baselineskip}\par\noindent{\normalsize\bfseries{#1}}\quad}

\def\elasticquad{\hspace{0em plus 1em}\relax}

\newcommand{\myq}[1]{\vspace{0.2\baselineskip}\par\noindent{\normalsize\emph{#1}}\quad}

\newcommand{\mydemotesturl}[1]{https://cerberus.cl.cam.ac.uk/cerberus?demo/#1}
\newcommand{\mytesturl}[1]{https://cerberus.cl.cam.ac.uk/cerberus?defacto/#1}
\newcommand{\mytestlink}[2]{\href{\mytesturl{#1}}{#2}}
\newcommand{\mylsttestlink}[1]{\mytestlink{#1}{\lstinline{#1}}}

%\newcommand{\mytestlink}[2]{\url{#2}} %\href{\mytesturl{#1}}{\lstinline{#2}}}
% macros for question-tooling generated latex
\newcommand{\myqtquestion}[2]{\myparagraph{#1. #2}}
%\newcommand{\myqtexample}[3]{\myexampleheader{#2}  \url{#3}\lstinputlisting{#1#2}}
\newcommand{\mylistingmargin}{5mm}
\newcommand{\myqtlinkexample}[4]{{\vspace*{-0.5\baselineskip}\par{\noindent\small\hspace*{\mylistingmargin}\lstinline{//} #4\lstinputlisting[showstringspaces=false,xleftmargin=\mylistingmargin,aboveskip=0mm]{tests-hand-edited/#2}\vspace*{-0.25\baselineskip}}}}

\newcommand{\myqtexample}[3]{\myqtlinkexample{#1}{#2}{#3}{{\mylsttestlink{#2}}}}

\newcommand{\myfooexamplename}[1]{\mylsttestlink{#1}}
\newcommand{\myfooexample}[3]{{\vspace*{-0.5\baselineskip}\par{\noindent\small\hspace*{\mylistingmargin}\lstinline{//} \mylsttestlink{#2}\lstinputlisting[showstringspaces=false,xleftmargin=\mylistingmargin,aboveskip=0mm]{#1/#2}\vspace*{0.25\baselineskip}\par}}}

\newcommand{\myfoolinkexample}[4]{{\vspace*{-0.5\baselineskip}\par{\noindent\small\hspace*{\mylistingmargin}\lstinline{//} #4\lstinputlisting[showstringspaces=false,xleftmargin=\mylistingmargin,aboveskip=0mm]{#1/#2}\vspace*{0.25\baselineskip}\par}}}

\newcommand{\mycerbexamplename}[2]{\mytestlink{#2}{\lstinline{#1}}}
\newcommand{\mycerbexample}[4]{{\vspace*{-0.5\baselineskip}\par{\noindent\small\hspace*{\mylistingmargin}\lstinline{//} \mycerbexamplename{#2}{#4}\lstinputlisting[showstringspaces=false,xleftmargin=\mylistingmargin,aboveskip=0mm]{#1/#2}\vspace*{0.25\baselineskip}\par}}}


\usepackage{alltt}

\makeatletter
\newcommand{\manuallabel}[2]{\def\@currentlabel{#2}\label{#1}}
\makeatother

\usepackage[scaled=0.82]{beramono}
\usepackage{listings}
\lstset{basicstyle=\renewcommand{\baselinestretch}{0.9}\tt\small}
%\lstset{basicstyle=\tt} 
\lstset{language=c}
%\lstset{numbers=left}
% \lstset{basicstyle=\ttfamily} 
% \lstset{basicstyle=\footnotesize\ttfamily} 
\lstset{keywordstyle=\bfseries}



%%%%%%%%%%%%%%%%%%%%%%%%%%%%%%%%%%%%%%%%%%%%%%%%%%%%%%%%%%%%%%

%\usepackage{tikz}
%\usepackage{pgfplots}

\newenvironment{tightenumerate}{%
\begin{enumerate}%
\setlength{\partopsep}{0pt}%
\setlength{\itemsep}{1pt}%
\setlength{\parskip}{0pt}%
\setlength{\parsep}{0pt}%
}{\end{enumerate}
}
\newenvironment{vtightenumerate}{%
\begin{enumerate}%
\setlength{\topsep}{0pt}%
\setlength{\partopsep}{0pt}%
\setlength{\itemsep}{1pt}%
\setlength{\parskip}{0pt}%
\setlength{\parsep}{0pt}%
}{\end{enumerate}
}

%\setlength{\leftmargini}{0mm}

\newenvironment{tightitemize}{
% \begin{itemize}
%   \setlength{\labelwidth}{0pt}
%   \setlength{\leftmargin}{0pt}
%   \setlength{\itemsep}{1pt}
%   \setlength{\parskip}{0pt}
%   \setlength{\parsep}{0pt}}{\end{itemize}
% }

% \newenvironment{verytightitemize}{
 \begin{itemize}
   \setlength{\itemsep}{0pt}
   \setlength{\parskip}{0pt}
   \setlength{\leftmargin}{0pt}
   \setlength{\leftmargini}{1.5mm}
   \setlength{\leftmarginii}{3mm}
   \setlength{\leftmarginiii}{4.5mm}
   \setlength{\labelwidth}{1.5mm}
   \setlength{\itemindent}{0mm}
   \setlength{\labelsep}{1.5mm}
   \setlength{\rightmargin}{0pt}
   \setlength{\topsep}{0pt}
   \setlength{\parsep}{0pt}}{\end{itemize}
 }

\newenvironment{verytightitemize}{
\begin{list}{\hspace*{-2mm}$\bullet$}{
  \setlength{\itemsep}{1pt}
  \setlength{\topsep}{2pt}
  \setlength{\parskip}{0pt}
  \setlength{\leftmargin}{2mm}
  \setlength{\labelwidth}{0pt}
%  \settowidth{\labelwidth}{$\bullet$}
%  \setlength{\itemindent}{0pt}
  \settowidth{\itemindent}{$\bullet$}
  \addtolength{\itemindent}{0mm}
  \setlength{\parsep}{0pt}}}{\end{list}
}

%%%%%%%%%%%%%%%%%%%%%%%%%%%%%%%%%
\newif\ifWithCharonData\WithCharonDatatrue
\WithCharonDatafalse
\newif\ifWithCharonSome\WithCharonSometrue
\WithCharonSomefalse


%%%%%%%%%%% typesetting of Charon output %%%%%%%%%%%%
\newcommand{\charontt}[1]{\texttt{#1}}
\newcommand{\charonZeroExitCodes}[2]{}
\newcommand{\charonNonZeroExitCodes}[2]{{\color{red}[*** exit codes #1 / #2 ***]}}
\newcommand{\charonSourcesMismatch}{{\color{red}SOURCES MISMATCH}}


% #1  tool instance name
% #2  tool command-line name
% #3  tool args
% #4  compile stderr    
% #5  compile and execute exit codes 
% #6  stdout            
% #7  stderr            
% #8  unused
% #9  execute signals
\newcommand{\charonSameOutcome}[9]{\noindent\textsc{#1:} %
 \ldots as above\\}
\newcommand{\charonSameModuloAddressesOutcome}[9]{\noindent\textsc{#1:} %
 \ldots as above (modulo addresses)\\%
%{}[#5/#8]#9\\\charontt{#4}\charontt{#6}\charontt{#7}%
}
\newcommand{\charonOutcome}[9]{\noindent\textsc{#1:} %
#5#9\\\charontt{#4}\charontt{#6}\charontt{#7}}

% #1  flavour (iso/defacto)
% #2  warnings
% #3  errors
% #4  return code
% #5  stdout
% #6  stderr
\newcommand{\charonExpectation}[6]{\noindent\textsc{#1:} %
\charontt{#2}\charontt{#3}\charontt{#5}}
% #1  target (iso/defacto)
% #2  expectation
\newcommand{\charonNewExpectation}[2]{\noindent\textsc{#1:} %
\charontt{#2}\\}
\newcommand{\charonTest}[2]{#2}
 
%%%% typesetting of examples, with and without code and Charon %%%%%% output %%
\newcommand{\myexamplefontsize}{\fontsize{8pt}{9pt}}
\newcommand{\myexamplename}[1]{\url{#1}}
%\newcommand{\myexampleheader}[1]{\medskip\par\noindent\textsc{Example} (\myexamplename{#1}):}
\newcommand{\myexampleheader}[1]{\medskip\par\noindent\small\textsc{#1}:}
%\newcommand{\myexamplecode}[1]{\verbatiminput{#1}}
\newcommand{\myexamplecode}[1]{\lstinputlisting{#1}}
\ifWithCharonData
\newcommand{\myexamplecharon}[1]{\input{#1}}
\else
\newcommand{\myexamplecharon}[1]{}
\fi
\newcommand{\indexTestName}[1]{\index{#1@\url{#1}}}

% just with the name  
\newcommand{\myjustnameexample}[1]{{%
\indexTestName{#1}%
\myexamplename{#1}}}

% with code and Charon output
\newcommand{\mycexample}[1]{{%
\indexTestName{#1}%
\myexamplefontsize%
\myexampleheader{#1}%
%\myexamplecode{examples/#1}%
\myexamplecode{../charon2/generated_tex/#1.src}%
%\par\noindent\myexamplefontsize%
%\myexamplecharon{examples/generated_tex/#1.out}%
\myexamplecharon{../charon2/generated_tex/#1.out}%
\par\noindent%
}}

% without code; with Charon output
\newcommand{\mynocodeexample}[1]{{%
\indexTestName{#1}%
\myexamplefontsize%
\myexampleheader{#1}%
\\%
%\myexamplecode{examples/#1}%
%\myexamplecharon{examples/generated_tex/#1.out}%
\myexamplecharon{../charon2/generated_tex/#1.out}%
\par\noindent%
}}

% with just code; with no Charon output
\newcommand{\myjustcodeexample}[1]{{%
\indexTestName{#1}%
\myexamplefontsize%
\myexampleheader{#1}%
\\%
%\myexamplecode{examples/#1}%
\myexamplecode{../examples/de_facto_memory_model/#1}%
%\myexamplecode{../charon2/generated_tex/#1.src}%
%\myexamplecharon{examples/generated_tex/#1.out}%
\par\noindent%
}}

% with just code; with no Charon output
\newcommand{\myjustOLDcodeexample}[1]{{%
\indexTestName{#1}%
\myexamplefontsize%
\myexampleheader{#1}%
\\%
%\myexamplecode{examples/#1}%
\myexamplecode{../charon2/tests/de_facto_memory_model/#1}%
%\myexamplecode{../charon2/generated_tex/#1.src}%
%\myexamplecharon{examples/generated_tex/#1.out}%
\par\noindent%
}}

% with code and Charon output
\newcommand{\myjustcodeNEWexample}[1]{{%
\indexTestName{#1}%
\myexamplefontsize%
\myexampleheader{#1}%
%\myexamplecode{examples/#1}%
\myexamplecode{#1}%
%\myexamplecode{../charon2/generated_tex/#1.src}%
%\par\noindent\myexamplefontsize%
%\myexamplecharon{examples/generated_tex/#1.out}%
%\myexamplecharon{../charon2/generated_tex/#1.out}%
\par\noindent%
}}


% with fake code and Charon output
\newcommand{\myfakecodeexample}[2]{{%
\indexTestName{#1}%
\indexTestName{#2}%
\myexamplefontsize%
\myexampleheader{#1}%
%\myexamplecode{examples/#2}%
\myexamplecode{../charon2/tests/de_facto_memory_model/#2}%
%\par\noindent\myexamplefontsize%
%\myexamplecharon{examples/generated_tex/#1.out}%
\myexamplecharon{../charon2/generated_tex/#1.out}%
\par\noindent%
}}


% pre-charon macro
\newcommand{\myexample}[1]{{\myexampleheader{#1}\myexamplefontsize\myexamplecode{examples/generated/#1}}}


% Cerberus Memory API
\usepackage{proof}
\DeclareMathOperator{\allocObject}{allocate\_object}
\DeclareMathOperator{\allocRegion}{allocate\_region}
\DeclareMathOperator{\Kill}{kill} % kill already exists
\DeclareMathOperator{\load}{load}
\DeclareMathOperator{\store}{store}
\DeclareMathOperator{\storeLocking}{store\_locking}
\DeclareMathOperator{\eqPtr}{eq\_ptrval}
\DeclareMathOperator{\nePtr}{ne\_ptrval}
\DeclareMathOperator{\diffPtr}{diff\_ptrval}
\DeclareMathOperator{\relOpPtr}{rel\_op\_ptrval}
\DeclareMathOperator{\eqOpPtr}{eq\_op\_ptrval}
\DeclareMathOperator{\ptrcastIval}{cast\_ival\_to\_ptrval} %intcast\_ptrval} %ptrFromInt  cast_int_to_ptr
\DeclareMathOperator{\intcastPtr}{cast\_ptrval\_to\_ival} %ptrcast\_ival}  %intFromPtr
\DeclareMathOperator{\combine}{combine\_prov}
\DeclareMathOperator{\opIval}{op\_ival}
\DeclareMathOperator{\eqIval}{eq\_ival}
\DeclareMathOperator{\ltIval}{lt\_ival}
\DeclareMathOperator{\leIval}{le\_ival}
\DeclareMathOperator{\offsetofIval}{offsetof\_ival}
\DeclareMathOperator{\arrayOffset}{array\_offset\_ptrval}
\DeclareMathOperator{\isoArrayOffset}{iso\_array\_offset\_ptrval}
\DeclareMathOperator{\memberOffset}{member\_offset\_ptrval}
\DeclareMathOperator{\wellAligned}{is\_well\_alligned}
\DeclareMathOperator{\validForDeref}{valid\_for\_deref}

% Types
\DeclareMathOperator{\type}{Type}
\DeclareMathOperator{\bytes}{[(Prov, Byte)]}
\DeclareMathOperator{\byte}{Byte}
\DeclareMathOperator{\bytemap}{ByteMap}
\DeclareMathOperator{\addr}{Addr}
\DeclareMathOperator{\memvalue}{MemValue}
\DeclareMathOperator{\mbytes}{[(Maybe\ Prov, Byte)]}
\DeclareMathOperator{\prov}{Prov}
\DeclareMathOperator{\allocMap}{AllocMap}

% Helper functions
\DeclareMathOperator{\dom}{dom}
\DeclareMathOperator{\newAlloc}{newAlloc}
\DeclareMathOperator{\sizeof}{sizeof}
\DeclareMathOperator{\dearray}{dearray}
\DeclareMathOperator{\alignof}{alignof}
\DeclareMathOperator{\abst}{abst}
\DeclareMathOperator{\repr}{repr}
\DeclareMathOperator{\fetch}{fetch}
\DeclareMathOperator{\update}{update}
\DeclareMathOperator{\readDevice}{readDevice}
\DeclareMathOperator{\storeDevice}{storeDevice}
\DeclareMathOperator{\getProv}{getProv}

% Constants
\newcommand{\Null}{\mathtt{null}} % null already defined
\newcommand{\provNone}{@\mathtt{empty}}
\newcommand{\true}{\mathtt{true}}
\newcommand{\false}{\mathtt{false}}
\newcommand{\readwrite}{\mathtt{readWrite}}
\newcommand{\readonly}{\mathtt{readOnly}}
\newcommand{\myobject}{\mathtt{object}}
\newcommand{\region}{\mathtt{region}}
\newcommand{\killed}{\mathtt{killed}}
\newcommand{\unspec}{\mathtt{unspec}}
\newcommand{\dyn}{\mathtt{dyn}}
\newcommand{\none}{\mathtt{none}}
\newcommand{\exposed}{\myt{\mathtt{exposed}}}
\newcommand{\unexposed}{\myt{\mathtt{unexposed}}}

\newcommand{\ruleOne}[5]{\minibox{[$\textsc{#2}$:\ $#1$] \vspace{0.0cm} \\
  \infer[] {#4  \rightarrow #5} {#3}}}
\newcommand{\ruleOneBreak}[5]{\minibox{[$\textsc{#2}$:\\ $#1$] \vspace{0.0cm} \\
  \infer[] {#4  \rightarrow #5} {#3}}}
\newcommand{\ruleTwo}[6]{\minibox{[$\textsc{#2}$:\ $#1$] \vspace{0.0cm} \\
  \infer[] {#5  \rightarrow #6} {\deduce {#4} {#3}}}}
\newcommand{\ruleThree}[7]{\minibox{[$\textsc{#2}$:\ $#1$] \vspace{0.0cm} \\
  \infer[] {#6  \rightarrow #7} {\renewcommand{\arraystretch}{1}\begin{array}{c}#3\\#4\\#5\end{array}}}}



%%%%%%%%%%%%%%%%%%%%%%%%%%%%%%%%%%%%%%%%%%%%%%%
\sloppy
\begin{document}

\ifWGfourteennumber
\fancypagestyle{firstpagestyle}{%
\fancyhf{} % clear all header and footer fields
\fancyhead[C]{ISO/IEC JTC1/SC22/WG14 N2311, 2018-11-09} % except the center
\renewcommand{\headrulewidth}{0pt}
\renewcommand{\footrulewidth}{0pt}}
\thispagestyle{plain}
\fi

\title[C provenance semantics: detailed semantics]{C provenance semantics: detailed
semantics (for PNVI-plain, PNVI address-exposed, PNVI address-exposed 
user-disambiguation, and PVI models)}


\authorsaddresses{}

\author{Peter Sewell}
\affiliation{
  \institution{University of Cambridge}         
}

 \author{Kayvan Memarian}
 \affiliation{
   \institution{University of Cambridge}            %% \institution is required
%   \country{UK}
 }
% %\authornote{with author1 note}          %% \authornote is optional;
%                                         %% can be repeated if necessary
% %\orcid{nnnn-nnnn-nnnn-nnnn}             %% \orcid is optional
% \affiliation{
% %  \position{Position1}
% %  \department{Department1}              %% \department is recommended
%   \institution{University of Cambridge}            %% \institution is required
% %  \streetaddress{Street1 Address1}
% %  \city{City1}
% %  \state{State1}
% %  \postcode{Post-Code1}
%   \country{UK}
% }
% %\email{first1.last1@inst1.edu}          %% \email is recommended
% 
% 
 \author{Victor B. F. Gomes}
 \affiliation{
   \institution{University of Cambridge}            %% \institution is required
%   \country{UK}
 }

\renewcommand{\shortauthors}{Sewell, Memarian, Gomes}

% %% Author with two affiliations and emails.
% \author{First2 Last2}
% \authornote{with author2 note}          %% \authornote is optional;
%                                         %% can be repeated if necessary
% \orcid{nnnn-nnnn-nnnn-nnnn}             %% \orcid is optional
% \affiliation{
%   \position{Position2a}
%   \department{Department2a}             %% \department is recommended
%   \institution{Institution2a}           %% \institution is required
%   \streetaddress{Street2a Address2a}
%   \city{City2a}
%   \state{State2a}
%   \postcode{Post-Code2a}
%   \country{Country2a}
% }
% \email{first2.last2@inst2a.com}         %% \email is recommended
% \affiliation{
%   \position{Position2b}
%   \department{Department2b}             %% \department is recommended
%   \institution{Institution2b}           %% \institution is required
%   \streetaddress{Street3b Address2b}
%   \city{City2b}
%   \state{State2b}
%   \postcode{Post-Code2b}
%   \country{Country2b}
% }
% \email{first2.last2@inst2b.org}         %% \email is recommended


%% Paper note
%% The \thanks command may be used to create a "paper note" ---
%% similar to a title note or an author note, but not explicitly
%% associated with a particular element.  It will appear immediately
%% above the permission/copyright statement.
%\thanks{with paper note}                %% \thanks is optional
                                        %% can be repeated if necesary
                                        %% contents suppressed with 'anonymous'


%% Abstract
%% Note: \begin{abstract}...\end{abstract} environment must come
%% before \maketitle command


%% \maketitle
%% Note: \maketitle command must come after title commands, author
%% commands, abstract environment, Computing Classification System
%% environment and commands, and keywords command.
\maketitle



\definecolor{myudicolor}{rgb}{0.5 0.0 0.8}
\newcommand{\myt}[1]{{\color{blue}#1}}
\newcommand{\myu}[1]{{\color{myudicolor}#1}}

\section{Introduction}

This note sketches a mathematical version of the
provenance-not-via-integer, address-exposed (PNVI-ae)
and provenance-not-via-integer, address-exposed, user-disambiguation (PNVI-ae-udi)
models, based on:
\begin{itemize}
\item the PNVI semantics in the POPL 2019
  paper~\cite{cerberus-popl2019}, here called PNVI-plain
\item the mathematical version  by Martin Uecker, 2019-01-18
\item the text summary by Jens Gustedt, 2019-01-22
\item email discussion on the C Memory Object Model study group mailing list
\end{itemize}
Changes for PNVI-ae from PNVI-plain are \myt{highlighted}.
Additional changes for PNVI-ae-udi are \myu{highlighted}. 

This should be read together with the two companion notes, one giving 
a series of example, and another giving detailed diffs to the
C standard text. 



This version of PNVI-* permits bytewise copy of a pointer
to an initially unexposed object, but leaves it marked as exposed.
Additional machinery may well be desirable for PNVI-ae and PNVI-ae-udi
to give programmers more control of the provenance of the results of
byte manipulations, and of what is left marked as exposed. The design of that machinery should
 ideally be based on the treatment of representation-byte-accessed
 pointer values by existing compiler alias analyses and optimisations.



\section{The PNVI-ae-udi, PNVI-ae, PNVI-plain, and PVI semantics}

The base PVI and PNVI-plain memory object semantics are manually typeset mathematics simplified from the
executable-as-test-oracle
Cerberus
mechanised Lem~\cite{Lem-icfp2014} source. We have removed most subobject details, function
pointers, and some options.  Neither the typeset models or
the Lem source consider linking, or pointers constructed via I/O
(e.g.~via
\texttt{\%p} or representation-byte I/O).

The memory object semantics can be combined with a semantics for the
thread-local semantics of the rest of C (expressed in Cerberus as a
translation from C source to the \emph{Core} intermediate language,
together with an operational semantics for Core) to give a complete
semantics for a large fragment of sequential C. 


For simplicity, we assume that pointer representations are the two's complement
representation of their addresses (and identical to the two's
complement representations of their conversions to sufficiently wide
integer types), assume NULL pointers have
address (and representation) $0$, and allow NULL pointers to be constructed from any
empty-provenance integer zero, not just integer constant expressions.

At present the model does not include the ISO semantics that makes all
pointers to an object or region indeterminate at the end of its
lifetime, and it permits equality comparison between pointers
irrespective of whether the objects of their provenances are live,
but it does permit pointer subtraction, relational comparison, array
offset, member offset, and casts to integer only for pointers to live
objects for which the address is within or one past the object
footprint.  These are all debatable choices.  One could instead check
only that the addresses are within or one past the original object
footprint (and not check the object is live), or go further towards a concrete-address view of pointer
values and not check that either.  Sketching out some of the options:
\begin{itemize}
\item \textsc{zombie-pointers-become-indeterminate}  For the current
  ISO semantics, at every storage instance lifetime end, the semantics
  should replace every pointer value with that provenance in the
  abstract-machine environment with the indeterminate value, and, for
  every memory footprint containing a pointer value with that
  provenance (that came from a single pointer value write), synthesise a 
  a write of the indeterminate value to that footprint.  With this,
  the live-object checks for equality, relational comparison,
  subtraction, array offset, member offset, and casts to integers all
  become moot. 
\item \textsc{zombie-pointers-allow-equality-only}  This is what the
  maths below details. 
\item \textsc{zombie-pointers-allow-all-in-bounds-arithmetic}
  For this, we would retain metadata for the bounds of lifetime-ended
  pointers and check against that for non-load/store operations.
\item \textsc{zombie-pointers-allow-all-arithmetic}  For this, we
  would remove the lifetime and bounds checks for non-load/store operations.
\item \textsc{all-pointers-allow-all-arithmetic}  This would make all
  the non-load/store operations operate just on abstract addresses,
  ignoring provenance and storage instance metadata.
\end{itemize}




\subsection{The memory object model interface}
In Cerberus, the memory object model is factored out from Core with a clean 
interface, roughly as in \citeN[Fig.~2]{Cerberus-PLDI16}.  
This provides 
functions for memory operations:
\begin{itemize}
\item $\allocObject$ %WAS \texttt{allocate\_static},
(for objects with automatic or static
storage duration, i.e.~global and local variables),  
\item $\allocRegion$ % WAS \texttt{allocate\_dynamic},
(for the results of \lstinline{malloc}, \lstinline{calloc}, and
\lstinline{realloc}, i.e.~heap-allocated regions),  
\item $\Kill$ (for lifetime end of both kinds of allocation),
\item $\load$, and
\item $\store$,
\end{itemize}
and for pointer/integer operations:
arithmetic, casts, comparisons,
offseting pointers by struct-member offsets, etc.
The interface involves types 
\texttt{pointer\_value} ($p$),
\texttt{integer\_value} ($x$),
\texttt{floating\_value}, 
and \texttt{mem\_value} ($v$), which are abstract as far as Core is
concerned.  
%
Distinguishing pointer and integer values gives more precise internal
types. %For example, when loading and storing via a pointer, one
%needs a case split that does not have to consider all 
%the ways that the pointer might have been constructed via integer
%operations. On the other hand, constraints (in symbolic-mode
%Cerberus) are entirely over
%unprovenanced integer values. 


In PNVI-ae, PNVI, and PVI, a provenance $\pi$ is either
 $@i$ where $i$ is a storage-instance ID, or the \emph{empty}
  provenance $\provNone$. 
%
\myu{In PNVI-ae-udi a provenance can also be a symbolic storage instance ID
 $\iota$ (iota), initially associated to two storage
 instance IDs and later resolved to one or the other.}
  %some other handling of adjacency-ambiguity}

A pointer value can either be $\Null$ or a pair $(\pi, a)$
of a provenance $\pi$ and address $a$.
In PNVI*, an integer value is simply a mathematical integer (within the appropriate range for the relevant C type), while in PVI, an
integer value is a pair $(\pi, n)$ of a provenance $\pi$ and a mathematical integer
$n$. 

Memory values are the storable entities, either a 
pointer, integer, floating-point, array, struct, or union value,
or $\unspec$ for unspecified values, each together with their C type. 



\section{The memory object model state}

In both PVI and PNVI*, a memory state is a pair $(A, M$). The  $A$ is a partial map from
storage-instance IDs to either $\killed$ or storage-instance metadata
$(n,\tau_{\text{opt}},a,f,k,t)$:
\begin{itemize}
\item size $n$, 
\item optional C type $\tau$ (or $\none$ for allocated regions), 
\item base address $a$, 
\item permission flag $f\mathord{\in}\{\readwrite,\,\readonly\}$, 
\item kind $k\mathord{\in}\{\myobject,\,\region\}$, and
\item \myt{for PNVI-ae and PNVI-ae-udi, a taint flag $t\mathord{\in}\{\unexposed,\,\exposed\}$}.
\end{itemize}
\myu{In PNVI-ae-udi, $A$ also maps all symbolic storage instance IDs $\iota$,
to sets of either one or two (non-symbolic) storage instance IDs.
One might also need to record a partial equivalence relation
over symbolic storage instance IDs, to cope with the pointer
subtraction and relational comparison cases where one learns that two
provenances are equal but both remain ambiguous, but that is debatable
and not
spelt out in this document.
}

The $M$ is a partial map from addresses to abstract bytes,
which are triples of a provenance $\pi$, either a byte $b$ or $\unspec$, and
an optional integer pointer-byte index $j$ (or $\none$).
The last is used in PNVI* to distinguish between loads of pointer
values that were written as whole pointer writes vs those that were written
byte-wise or in some other way.




\subsection{Mappings between abstract values and representation
abstract-byte sequences}

The $M$ models the memory state in terms of low-level abstract
bytes, but   \texttt{store} and \texttt{load} take and return
the higher-level memory values. We relate the two with functions
$\repr(v)$, mapping a memory value to a list of abstract bytes, and
$\abst(\tau, bs)$, mapping a list of abstract bytes $bs$ to its
interpretation as a memory value with C type $\tau$.


The $\repr(v)$ function is defined by induction over the structure of its memory value
parameter and returns a list of $\sizeof(\tau)$ abstract bytes, where
$\tau$ is the C type of the parameter. The base cases are values with
scalar types (integer, floating and pointers) and unspecified values.
%
For an unspecified value of type $\tau$, it returns a list with abstract bytes of the form
$(\provNone, \unspec, \none)$.
%
Non-null pointer values are represented with lists of abstract bytes that
each have the provenance of the pointer value, 
the appropriate part of the two's complement encoding of the address,
and the $0..\sizeof(\tau)-1$ index of each byte. 
Null pointers are represented with lists of abstract bytes of the form $(\provNone, 0,
\none)$. 
In PVI, integer values are represented similarly to pointer values
except that the third component of each abstract byte is $\none$.
In PNVI*, integer values are represented by lists of abstract bytes,
with each of their first components always the empty provenance,  and
each of their third components again $\none$.
%
Floating-point values are similar, in all the models, except that the provenance of the
abstract bytes
is always empty.
%
%
For array and struct/union values the function is inductively
applied to each subvalue and the resulting byte-lists concatenated.
%into
%a single list.
The layout of structs and unions follow an
implementation-defined ABI, with padding bytes like those
of unspecified values.


The $\abst(\tau, bs)$ function is defined by induction over
$\tau$. The base cases are again the scalar types. For these, $\sizeof(\tau)$
abstract bytes are consumed from $bs$ and a scalar memory value is
constructed from their second components: if any abstract byte has an
$\unspec$ value, an unspecified value is constructed; otherwise,
depending on $\tau$, a pointer, integer or floating-point value is
constructed using the two's complement or floating-point encoding.
%
For pointers with address $0$, the provenance is empty. For non-$0$ pointer
values and integer values, in PVI the provenance is constructed as
follows: if 
%all the abstract bytes have the same provenance, 
at least one abstract byte has non-empty provenance and all others
have either the same or empty provenance, 
that
provenance is taken, otherwise the empty provenance is taken.
In PNVI*, when constructing a pointer value, 
if the third components of the bytes all carry the appropriate
index, and all have the same provenance (which will be guaranteed if
pointer types all have the same size), the provenance of the result is that provenance. Otherwise, the $A$ part of the memory state is examined to find whether a live
storage instance exists with a footprint containing the pointer value that
is being constructed. If so, in PNVI-plain, its storage instance ID
is used for the provenance of the pointer value, otherwise the empty
provenance is used.
\myt{In PNVI-ae and PNVI-ae-udi, when constructing a pointer value, if $A$ has to
be examined then, matching the relevant integer-to-pointer cast
semantics below, the storage instance must have been exposed,
otherwise  the result have the empty provenance.}
\myu{In PNVI-ae-udi, if there are two such live storage instances, with
IDs $i_1$ and $i_2$, the resulting pointer value is given a fresh
symbolic storage instance ID $\iota$, and $A$ is updated to map
$\iota$ to $\{i_1,i_2\}$.  This can only happen if the two storage
instances are adjacent and the address is one-past the first and at the
start of the second.}
For array/struct types, $abst()$ recurses on %calls itself for each subtype on
the progressively shrinking list of abstract bytes.




\subsection{Memory operations}

The successful semantics of memory operations is expressed as a transition
relation between memory states, with transitions labelled by the
operation (including its arguments) and return value:
\[
(A,M)\xrightarrow{\textsc{label}} (A',M')
\]
For example, the transitions 
\[
(A,M) \xrightarrow{\load(\tau,p)=v} (A',M')
\]
describe the semantics of a $\load(\tau,p)$ in memory state $(A,M)$,
returning value $v$ and with resulting memory state $(A',M')$. 
%
The semantics also defines when each operation flags an out-of-memory
(OOM) or undefined behaviour (UB) in a memory state $(A,M)$. 

\myparagraph{Storage instance creation}
When a new storage instance is created, either with 
$\allocRegion$ 
(for the results of \lstinline{malloc}, \lstinline{calloc}, and
\lstinline{realloc}, i.e.~heap-allocated regions),
or with 
 $\allocObject$ 
(for objects with automatic or static
storage duration, i.e.~global and local variables), in
 non-\texttt{const} and \texttt{const} variants:
a fresh storage-instance ID $i$ is chosen;
an address $a$ is chosen from
  $\newAlloc(A, al, n)$, defined to be the set of addresses of blocks of size $n$
  aligned by $al$ that do not overlap with $0$ or any other allocation
  in $A$;
  and the pointer value $p = (@i, a)$ is returned.
In all three cases the storage-instance metadata $A$ is updated with a
new record for $i$, \myt{and this is initially marked as $\unexposed$}. 
In the $\allocObject$ case the size $n$ of the allocation is the
representation size of the C type $\tau$.
In the $\allocRegion(al, \tau, \readonly(v))$  case, the last of the
 three rules, the memory $M$ is updated to contain the representation
 of $v$ at the addresses $a..a+\sizeof(\tau)-1$. 
\[\renewcommand{\arraystretch}{2}\begin{array}{@{}l@{}} %{\hspace*{33mm}}r@{}}
    \ruleTwo
      {\allocRegion(al, n)=p}
      {label}
      {i \notin \dom(A) \qquad a \in \newAlloc(A, al, n)}
      {p = (@i, a)}
      {A, M} {A[i \mapsto (n, \none, a, \readwrite, \region, \unexposed)], M}
\\
    \ruleTwo
      {\allocObject(al, \tau, \readwrite)=p}
      {label}
      {i \notin \dom(A) \qquad a \in \newAlloc(A, al, n)}
      {n = \sizeof(\tau) \qquad p = (@i, a)}
      {A, M} {A(i \mapsto (n, \tau, a, \readwrite, \myobject, \unexposed)), M}
\\
    \ruleTwo
      {\allocObject(al, \tau, \readonly(v))=p}
      {label}
      { i \notin \dom(A) \qquad a \in \newAlloc(A, al, n)}
      {n = \sizeof(\tau) \qquad p = (@i, a)}
      {A, M} {A(i \mapsto (n, \tau, a, \readonly, \myobject, \unexposed)),
              M([a..a + n-1] \mapsto \repr(v))}
\end{array}
\]


\myparagraph{Storage instance lifetime end}  When the storage instance of
      a pointer value $(@i,a)$ 
is killed, either by a \texttt{free()} for a heap-allocated region or at
the end of lifetime of an object with automatic storage duration, the
storage-instance metadata $A$ of storage instance $i$ is updated to record that $i$ has been killed.
\[
\ruleTwo
      {\Kill(p, k)}
      {label}
      {p = (@i, a) \qquad k=k'}
      {A(i) = (n, \_, a, f, k', \myt{\_})}
      {A, M} {A(i \mapsto \killed), M}
\]



\myparagraph{Load} To load a value $v$ of type $\tau$ from a pointer value $p=(@i,a)$, there must
      be a live storage instance for $i$ in $A$, the footprint of
      $\tau$ at $a$ must be within the footprint of that allocation,
      and the value $v$ must be the abstract value obtained from the
      appropriate memory bytes from $M$.
\[\ruleThree
      {\load(\tau, p)=v}
      {label}
      {p = (@i, a) \qquad A(i) = (n, \_, a', f, k, \myt{\_})}
      {{}[a..a+\sizeof(\tau)-1] \subseteq [a'..a' + n-1]{}}
      {v = \abst(\tau, M[a..a+\sizeof(\tau)-1])}
      {A, M} {A, M}
\]
\myt{For PNVI-ae and PNVI-ae-udi, if the recursive-on-$\tau$ computation of 
  $\abst(\tau, M[a..a+\sizeof(\tau)-1])$
  involves a call of $\abst$ at any non-pointer scalar type for a
  region of $M$ including an abstract byte with non-empty provenance,
  and the corresponding storage instance is live, it is marked as
  exposed.
  This applies e.g.~for reads of pointer values via \lstinline{char*}
  pointers, and for union type punning reads at \lstinline{uintptr_t}
  of pointer values.}
  
\myparagraph{Store} To store a value $v$ of type $\tau$ to a pointer value $p=(@i,a)$, there must
      be a live storage instance for $i$ in $A$, which must be
      writable, and the footprint of
      $\tau$ at $a$ must be within the footprint of that allocation.
      The memory $M$ is updated with the representation bytes of the
      value $v$. 
\[
\ruleTwo
      {\store(\tau, p, v)}
      {label}
      {p = (@i, a) \qquad A(i) = (n, \_, a', \readwrite, k, \myt{\_})}
      {{}[a..a+\sizeof(\tau)-1] \subseteq [a'..a' + n-1]{}}
      {A, M} {A, M([a..a+\sizeof(\tau)-1] \mapsto \repr(v))}
\]


\myu{For PNVI-ae-udi, the $\Kill$, $\load$, and $\store$  rules above must be adapted.  If $p=(\iota,a)$
      and $A(\iota)=\{i\}$, the other premises and conclusion of the
      appropriate above rule apply.
      If $A(\iota)=\{i_1,i_2\}$ and the premises are
      satisfied for one of the two, say $i_j$, the rest of the rule applies
      except that in the
      final state $A$ is additionally updated to map $\iota$ to $\{i_j\}$.}



The memory operations flag out-of-memory (OOM) and undefined behaviour
(UB) as follows:
\begin{center}
\small
\begin{tabular}{@{}lll@{}}\hline
\multicolumn{3}{@{}l@{}}{$\allocRegion(al, n)$ / $\allocObject(al, \tau, \mathtt{readwrite})$
/ $\allocObject(al, \tau, \readonly(v))$:} \\
& OOM out of memory & if $\newAlloc(A, al, n)=\{\} $ or $\newAlloc(A, al, \sizeof(\tau))=\{\} $\\\hline
\multicolumn{3}{@{}l@{}}{$\load(\tau, p)$ / $\store(\tau, p, v)$ / $\Kill(p)$:}\\
& UB null pointer & if $p = \Null $\\
& UB empty provenance & if $p = (\provNone, a) $\\
& UB killed provenance & if $p = (@i, a)$ and $A(i)=\killed$\\\hline
%P I made the above dead provenance up. It cannot happen with pointer-lifetime-zap?
\multicolumn{3}{@{}l@{}}{$\load(\tau, p)$ / $\store(\tau, p, v)$:}\\
& UB out of bounds & if
      $p=(@i,a)$, $A(i) = (n, \_, a', f, k, \myt{\_})$, and
      ${}[a..a+\sizeof(\tau)-1] \not\subseteq [a'..a' + n-1] {} $ \\\hline
\multicolumn{3}{@{}l@{}}{$\store(\tau, p, v)$:}\\
& UB read-only & if  $p=(@i,a)$ and $A(i) = (n, \_, a', \readonly, k, \myt{\_})$\\\hline
\multicolumn{3}{@{}l@{}}{$\Kill(p)$:}\\
& UB non-alloc-address & if
      $p=(@i,a)$, $A(i) = (n, \_, a', f, k, \myt{\_})$, and $a\not=a'$ \\\hline
\end{tabular}
\end{center}
% 

\myu{For PNVI-ae-udi, the rules above must be adapted. In the case where $p=(\iota,a)$
      and $A(\iota)=\{i\}$, the semantics is exactly as for $p=(i,a)$,
      while if $A(\iota)=\{i_1,i_2\}$, one has UB only if the
      conditions above apply to both $i_1$ and $i_2$.}






\subsection{Pointer / Integer operations}


\myparagraph{Pointer subtraction} Pointers $p=(@i,a)$ and
      $p'=(@i',a')$ can be subtracted if they have the same provenance
      ($i=i'$), there is a live storage instance for $i$ in $A$, and
      both $a$ and $a'$ are within or one-past the footprint of that
      allocation (in ISO C the last will always hold, otherwise UB
      would have been flagged in earlier pointer
      arithmetic). Otherwise UB.
The result is the numerical difference $a-a'$ divided by
      $\sizeof(\dearray(\tau)))$, where 
$\dearray(\tau)$ returns $\tau$ if it is not an array type,
and otherwise returns its element type.
Note that this disallows subtraction for which one or both arguments
are null pointers, which is the ISO semantics but may be 
a debatable choice.

%
This rule is stated for PNVI \myt{and PNVI-ae}, returning pure integer.
For PVI,  $\diffPtr$ constructs the same integer
but with $\provNone$ provenance.
\myu{For PNVI-ae-udi, because subtraction of pointers with different
provenance should be UB:
\begin{itemize}
\item  if both the two pointers have either a provenance $@i$
  (resp.~$@i'$) or a symbolic storage instance ID $\iota$
  (resp.~$\iota'$) mapped by $A$ to a singleton $\{i\}$
  (resp.~$\{i'\}$), then $i=i'$, otherwise UB.
\item  if one of the two pointers has a symbolic storage
instance ID $\iota$, mapped by $A$ to $\{i_1,i_2\}$, while the other
either has a provenance $@i'$ or an $\iota'$ mapped to a singleton
$\{i'\}$,
then $i'$ must be either $i_1$ or $i_2$, and $\iota$ is resolved to
that in the $A$ of the final state. Otherwise UB.

\item If both pointers are ambiguous, say mapped to $\{i_1,i_2\}$ and $\{i'_1,i'_2\}$,
then if those two sets share exactly one element which satisfies the other rule
preconditions, both symbolic storage instance IDs are resolved to
that. Otherwise UB. 
\item If both pointers are ambiguous and 
those sets share two elements that satisfy the other
conditions (which we believe can only happen if the addresses are
equal), then subtraction is permitted but the symbolic storage
instance IDs are left unresolved. Otherwise UB.

For example, suppose \lstinline{p} and \lstinline{q} have been produced by separate casts
 from an integer which is ambiguously one-past one allocation and at
 the start of another. Then after \lstinline{p-q} or \lstinline{p<q} we know they must have been
 the same provenance, but we still don't know which. 
 (Alternatively, we could change the semantics to record an identity
 relation over  symbolic storage
instance IDs, and additional modifications to the rules below beyond what is in
this draft, but that seems to be unwarranted complexity). 
\end{itemize}
}
\[
\ruleTwo
      {\diffPtr(\tau, p, p')=x}
      {label}
      {p = (@i, a) 
        \qquad p' = (@i', a') \qquad i=i' \qquad A(i) = (n, \_, \hat{a}, f, k, \myt{\_})}
      {x = (a - a') / \sizeof(\dearray(\tau))
%        \qquad \hat{a} \le a \le \hat{a} + n \qquad \hat{a} \le a' \le \hat{a} + n}
        \qquad a \in [\hat{a}..\hat{a}+n] \qquad a' \in [\hat{a}..\hat{a}+n]{}}
      {A, M} {A, M}
\]

\myparagraph{Pointer relational comparison}
Pointers $p=(@i,a)$ and
      $p'=(@i',a')$ can be compared with a relational operator
      (\verb+<+, \verb+<=+, etc.) if they have the same provenance
      ($i=i'$).  The result is the boolean result of the mathematical
      comparison of $a$ and $a'$.
To make this analogous to pointer subtraction, we also require (though
this is debatable) that 
there is a live storage instance for $i$ in $A$, and
      both $a$ and $a'$ are within or one-past the footprint of that
      allocation. Otherwise UB. 
Note that this disallows relational comparison against null pointers;
a debatable choice.
      \myu{For PNVI-ae-udi, this has to be adapted in much the same way as the
      pointer subtraction rule above.}
\[
\ruleTwo
      {\relOpPtr(p,p',\mathit{op})=b}
      {label}
      {p = (@i, a) \qquad p' = (@i', a') \qquad i=i' \qquad A(i) = (n, \_, \hat{a}, f, k, \myt{\_})}
      {b = \mathit{op}(a, a') \qquad \qquad a \in [\hat{a}..\hat{a}+n]
        \qquad a' \in [\hat{a}..\hat{a}+n]{}\qquad \mathit{op} \in \{\le, <, >, \ge\}}
      {A, M} {A, M}
\]
Relational comparison is used in practice between pointers to
different objects.   A variant which would allow that, which we call
\textsc{allow-inter-object-relational-operators}
\textsc{true}, removes the $i=i'$ test above and (in the
\textsc{zombie-pointers-become-indeterminate} and 
\textsc{zombie-pointers-allow-equality-only} variants) additionally
checks that $i'$ maps to a live object with in-range address.

\myparagraph{Pointer equality comparison}
Pointers $p$ and $p'$ can always be compared with an equality operator
      (\verb+=+, \verb+!=+). 
       The result is true if they are either both null or both
      non-null and have the same provenance and
      address; nondeterministically either $a=a'$ or $\texttt{false}$
      if they are both non-null and have different provenances; and
      false otherwise.
\myu{For PNVI-ae-udi, because equality comparison is permitted (without
      UB) irrespective of the provenances of the pointers, if the two pointers both have determined single
      provenances after looking up any symbolic IDs in A, this should
      give $\mathtt{true}$, otherwise the middle (nondeterministic)
      clause should apply.  The final $A$ should not resolve any
      symbolic IDs.}
 \[
\ruleOne
      {\eqOpPtr(p,p')=b}
      {label}
      {\begin{cases}
       b=\mathtt{true} & \text{if $p=p'$} \\
       b \in \{(a=a'),\mathtt{false}\} & \text{if $p=(\pi,a)$, $p'=(\pi',a')$, and $\pi\not=\pi'$}\\
       b=\mathtt{false} & \text{otherwise} 
    \end{cases}
}
      {A, M} {A, M}
\]
Note that the above nondeterminism appears to be necessary to admit the
observable behaviour of current compilers, but a simpler
provenance-oblivious semantics is arguably desirable:
 \[
\ruleOne
      {\eqOpPtr(p,p')=b}
      {label}
      {\begin{cases}
       b=\mathtt{true} & \text{if $p=p'=\texttt{null}$}\\
       b=\mathtt{true} & \text{if $p=(\pi,a)$, $p'=(\pi',a')$, and $a=a'$}\\
       b=\mathtt{false} & \text{otherwise} 
    \end{cases}
}
      {A, M} {A, M}
\]
We call these two options 
\textsc{pointer-equality-provenance-nondet} \textsc{true} and \textsc{false}.



\myparagraph{Pointer array offset} Given a pointer $p$ at C type
      $\tau$, the result of offsetting $p$ by integer $x$ (either by array
      indexing or explicit pointer/integer addition) is as follows,
where $x=n$ in PNVI*, or   $x=(\pi', n)$ in PVI.
For the operation to succeed, $p$ must be some non-null $(@i,a)$.
Then there must be a live storage instance for $i$, and the numeric
      result of the addition of $a+n*\sizeof(\tau)$ must be within or
      one-past the footprint of that storage instance.
      Otherwise the operation flags UB.
      \myu{For PNVI-ae-udi, if $p$ is ambiguous (i.e., $p=(\iota,a)$ and
      $A(\iota)=\{i_1,i_2\}$ then if $x$ is non-zero this should only
      be defined behaviour for (at most) one of the two, and then
      $\iota$ should be resolved to that one in the final state.  If
      $x=0$ it does not resolve the ambiguity. }
\[
\isoArrayOffset(A, p, \tau, x) =
    \begin{cases}
      (@i, a') &% %\text{if} 
\begin{array}{@{}l@{}}
\text{if } p = (@i, a)   \text{ and}\\
a'=a + n*\sizeof(\tau) \text{ and}\\
A(i) = (n'', \_, a'', \_, \_, \myt{\_}) \text{ and}\\
a'\in [a''..a''+n'']
\end{array}\\
      \text{UB: out of bounds}
        &\text{if all except the last conjunct}\\
        &\text{ above hold}\\
      \text{UB: empty prov}
        &\text{if $p=(\provNone,a)$}\\
      \text{UB: killed prov}
        &\text{if $p=(@i,a)$ and $A(i)=\killed$}\\  
      \text{UB: null pointer}
        &\text{if $p = \Null$} \\
    \end{cases}
\]



\myparagraph{Pointer member offset} Given a non-null pointer $p$ at C type
        $\tau$,
        which points to the start of a struct or union type object
(ISO C suggests this has to exist, writing \emph{``The value is that of the named
member of the object to which the first expression points''})
with a member $m$,
        if $p$ is $(\pi,a)$,
the result of offsetting the pointer to member $m$ has the same
        provenance $\pi$ and the suitably offset $a$.

If $p$ is null, the result is a pointer with empty provenance and the
        integer offset of $m$ within $\tau$'s representation
        (this is de facto C behaviour, in the sense that the GCC torture tests rely
        on it; it does not exactly match ISO C).

For the first case, 
$p$ should point to the start of an object of type $\tau$, with UB
otherwise, but without a
subobject-aware effective-type semantics, we cannot check that
here.  Instead, we just check that there is a live storage instance of
$p$'s provenance such that the resulting address is within or one-past
its 
a footprint. That makes this analogous to pointer
array offset. 
\[
    \memberOffset(p, \tau, m) =
    \begin{cases}
      (\pi, a'),
& \begin{array}{@{}l@{}}
\text{if } p = (@i, a)   \text{ and}\\
a'=a + \offsetofIval(\tau, m) \text{ and}\\
A(i) = (n'', \_, a'', \_, \_, \myt{\_}) \text{ and}\\
a'\in [a''..a''+n'']
\end{array}\\
      (\provNone,\offsetofIval(\tau, m)),
        &\text{if $p = \Null$.}
    \end{cases}
\]



%
%

\myparagraph{Casts (PNVI-plain)}
In PNVI-plain, a cast of a pointer value $p$  to an integer value (at type $\tau$)
just converts null pointers to zero and non-null pointer values to the
address $a$ of the pointer, if that is representable in $\tau$,
otherwise flagging UB.  The provenance of the pointer is discarded.
At present we require that the object is live and that its address is
within bounds. 
\[ \begin{array}{@{}r@{\,}c@{\,}l@{}}
\intcastPtr(\tau, p) &=&
    \begin{cases}
       0,
        &\text{if } p = \Null; \\
       a,
& \begin{array}[t]{@{}l@{}}
\text{if } p = (@i, a)   \text{ and}\\
A(i) = (n'', \_, a'', \_, \_, \myt{\_}) \text{ and}\\
a\in [a''..a''+n''] \text{ and } a \in \mathtt{value\_range}(\tau) \\
\end{array}\\
      \text{UB},
        &\text{otherwise}
    \end{cases}
\end{array}
\]

In PNVI-plain, an integer-to-pointer cast of $0$ returns the null pointer.
For a non-$0$ integer $x$, casting to a pointer to $\tau$, if there is a storage instance $i$ in the
current memory model state $(A,M)$ for which the 
address of the pointer would be properly within the
footprint of the storage instance, it returns a pointer $(@i,x)$ with
the provenance of that storage instance.  (The ``properly within''
prevents the one-past ambiguous case.)  If there is no such storage
instance, it returns a pointer with empty provenance. 
\[  \begin{array}{@{}l@{}}
    \ptrcastIval(\tau, x) \\
\multicolumn{1}{@{}r@{}}{ \hspace*{10mm}=    \begin{cases}
      \Null,
        &\text{if $x = 0$} \\
      (@i,x),
%        &\text{if $A(i) = (n, \_, a, f, k, \myt{\_})$ and $[x..x+\sizeof(\tau)-1] \subseteq [a..a+n-1]$}\\
        &\text{if $A(i) = (n, \_, a, f, k, \myt{\_})$ and $x \in [a..a+n-1]$}\\
      (\provNone,x),
        &\text{if there is no such $i$}
    \end{cases}}
\end{array} 
\]
%It is debatable whether we should check the just the address $x \in
%[a..a+n-1]$, or that the footprint of an instance of tau is there.
%But the latter one make some one-past casts and back invalid.

\myt{
\myparagraph{Casts (PNVI-ae)}
In PNVI-ae, the result of a cast of a pointer value $p$  to an
integer value is exactly as in PNVI-plain.  In addition, 
for a cast of pointer value $p=(@i,a)$ with provenance $@i$, 
where $A(i) = (n,\tau_{\text{opt}},a,f,k,t)$ is the storage instance
metadata for $i$, 
the memory state
$(A,M)$ is updated to 
$(A(i\mapsto (n,\tau_{\text{opt}},a,f,k,\exposed)),M)$ to mark the
that storage instance as exposed. 
}

In PNVI-ae, an integer-to-pointer cast of $0$ returns the null pointer.
For a non-$0$ integer $x$, casting to a pointer to $\tau$, if there is a storage instance $i$ in the
current memory model state $(A,M)$ for which 
the address of the pointer would
be properly within the
footprint of the storage instance, \myt{and storage instance $i$ is
exposed}, it returns a pointer $(@i,x)$ with
the provenance of that storage instance.  If there is no such storage
instance, it returns a pointer with empty provenance. 
\[  \begin{array}{l}
    \ptrcastIval(\tau, x)\\
\multicolumn{1}{@{}r@{}}{ \hspace*{10mm} =    \begin{cases}
      \Null,
        &\text{if $x = 0$} \\
      (@i,x),
%        &\text{if $A(i) = (n, \_, a, f, k, \myt{\exposed})$ and $[x..x+\sizeof(\tau)-1] \subseteq [a..a+n]$}\\
        &\text{if $A(i) = (n, \_, a, f, k, \myt{\exposed})$ and $x \in
      [a..a+n-1]$}\\
      (\provNone,x),
        &\text{if there is no such $i$}
    \end{cases}}
\end{array}
\]

\myu{
\myparagraph{Casts (PNVI-ae-udi)}
In PNVI-ae-udi, a cast of a pointer value $p$ to an integer is just like
PNVI-ae.

Unlike PNVI-ae, PNVI-ae-udi permits a cast of a one-past pointer to
integer and back to recover the original provenance, replacing the
integer-to-pointer check that $x$ is properly within the footprint of
the storage instance by a check that it is properly within or one-past:
\[  \begin{array}{l}
    \ptrcastIval(\tau, x)\\
\multicolumn{1}{@{}r@{}}{ \hspace*{10mm} =    \begin{cases}
      \Null,
        &\text{if $x = 0$} \\
      (@i,x),
%        &\text{if $A(i) = (n, \_, a, f, k, \myt{\exposed})$ and $[x..x+\sizeof(\tau)-1] \subseteq [a..a+n]$}\\
        &\text{if $A(i) = (n, \_, a, f, k, \myt{\exposed})$ and $x \in
      [a..a+\myu{n}]$}\\
      (\provNone,x),
        &\text{if there is no such $i$}
    \end{cases}}
\end{array}
\]
But then a PNVI-ae-udi cast of an integer value to a pointer
can create a pointer with ambiguous provenance (as in the definition of $\repr$)
: 
if it could be within or one-past two live storage instances, with
IDs $i_1$ and $i_2$, 
and both storage instances have been marked as exposed,
the resulting pointer value is given a fresh
symbolic storage instance ID $\iota$, and $A$ is updated to map
$\iota$ to $\{i_1,i_2\}$.  This can only happen if the two storage
instances are adjacent and the address is one-past the first and at the
start of the second.
}


\myparagraph{Casts (PVI)}
\[  \begin{array}{rcl}
    \ptrcastIval(\tau, x) &=&
    \begin{cases}
      \Null,
        &\text{if $x = (\provNone, 0)$} \\
      (\pi,n),
        &\text{otherwise, where $x = (\pi, n)$}
    \end{cases}
\\[4mm]
\intcastPtr(\tau, p) &=&
    \begin{cases}
      (\provNone, 0),
        &\text{if } p = \Null; \\
      (\pi, a),
        &\text{if } p = (\pi, a) \text{ and } a \in \mathtt{value\_range}(\tau) \\
      \text{UB},
        &\text{otherwise}
    \end{cases}
  \end{array}\]

\myparagraph{Integer operations (PVI)}
In PVI one also has to define the provenance results of all the other
operations returning integer values.  Below we do so for the basic
operations, though this would also be needed for all the
integer-returning library functions.  Most would give integers with
empty provenance.  One might or might not also want to require that the objects of
those provenances are live. 
\[  \begin{array}{rcl}
   \pi \oplus \pi' &=&
    \begin{cases}
      \pi,           &\text{if $\pi = \pi' \text{ or } \pi' = \provNone$;} \\
      \pi',          &\text{if $\pi = \provNone$;} \\
      \provNone,     &\text{otherwise.}
    \end{cases}
    \\[6mm]
    \opIval(op, (\pi, n), (\pi', m)) &=& (\pi \oplus \pi', op(n, m)),
      \text{ where } op \in \{+, *, /, \%, \&, |, \wedge\} \\
 \opIval(-, (\pi, n), (\pi', m)) &=&
    \begin{cases}
      (\provNone,n-m),           &\text{if $\pi = @i$  and 
    $\pi' = @i'$, whether $i=i'$ or not;} \\
      (@i,n-m),         &\text{if $\pi=@i$ and $\pi' = \provNone$;} \\
      (\provNone,n-m),     &\text{if $\pi=\provNone$.}
    \end{cases}
\\[6mm]
    \eqIval((\pi, n), (\pi', m)) &=& (n = m) \\
    \ltIval((\pi, n), (\pi', m)) &=& (n < m) \\
    \leIval((\pi, n), (\pi', m)) &=& (n \le m)
  \end{array}\]


\myt{
\subsection{No-expose annotation}
For PNVI-ae and PNVI-ae-udi,
to permit implementations, e.g.~of \lstinline{memcpy}-like functions,
to operate on representation bytes but without needlessly leaving all
the storage instances that were pointed to in those bytes exposed,
we envisaged some ``\lstinline{no-expose}'' annotation that
users could apply to such code.   
%
But now it's not so clear how that could work.
We can turn off exposure during execution of annotated code easily
enough (though Jens points out that this might not be the right thing
for code which is passed a function pointer).
But if the user-memcpy code
copies bytes via a \lstinline{char *} pointer, then the resulting
abstract types in memory still have empty provenance (because we're
not tracking provenance via the intervening integer values), so when
a pointer value is read (after the user-memcpy) from the copy, it will
still get empty provenance.  
}

\subsection{Provenance of other operations}
In addition to the operations defined above, 
some operations are desugared/elaborated to simpler
expressions by the Cerberus pipeline. 
Their PVI results have
provenance  as follows; their PNVI* results are the same except that
there integers have no provenance:
\begin{tightitemize}
  \item
    the result of address-of (\lstinline{&}) has the provenance of the
    object associated with the lvalue, for non-function-pointers, or
  empty for function pointers.
  \item
    prefix increment and decrement operators follow the corresponding pointer or
    integer arithmetic rules.
  \item
    the conditional operator has the provenance of the second or third
    operand as appropriate;
    simple assignment has the provenance of
    the expression; compound assignment follows the pointer or integer
    arithmetic rules; the comma operator has the provenance of
    the second operand.
  \item
    integer unary \lstinline{+}, unary \lstinline{-}, and
    \lstinline{~} operators preserve the original
    provenance; logical negation \lstinline{!} has a value with empty provenance.
  \item
    \lstinline{sizeof} and \lstinline{_Alignof} operators give values with
    empty provenance.
  \item
    bitwise shifts has the provenance of their first operand.

\item
Jens Gustedt highlights that atomic operations have their own specific provenance
properties, not yet discussed here.



\end{tightitemize}

\subsection{Provenance for other library functions}
TBD

\subsection{Provenance checks and race conditions}




\bibliography{csembib-fullnames}






\end{document}
